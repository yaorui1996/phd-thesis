% !TeX root = ./PhDThesis.tex

\begin{resolution}

  基于囚禁离子的系统是最有希望实现大规模量子计算及量子模拟的平台之一。梅全鑫同学在攻读博士期间搭建了一套囚禁Yb-171离子的实验平台,并利用该囚禁离子进行了相关的量子操控,开展了富有成果的实验研究。论文取得如下创新性成果:
  \begin{enumerate}
    \item 在常温系统中实现多达60个离子的稳定囚禁,并实现了自动化的多离子载入、高保真度的初态制备、基于CCD的离子状态测量等多离子量子模拟和量子计算的必备工具。
    \item 利用2至16离子构成的离子链作为量子信息处理的载体模拟自旋-声子耦合的Rabi-Hubbard(RH)模型,实验上通过调节RH模型的自旋-声子耦合强度,在不同系统规模中研究了自旋-自旋关联序参量的变化,观测到了量子相变。
    \item 在小系统中验证了量子动力学过程,与理论计算结果相符。并在16离子的系统中探究了该系统的动力学演化,希尔伯特空间维度超过2的57次方,超越经典计算机的模拟能力。
  \end{enumerate}


  姚睿同学的论文写作规范,逻辑严密,数据详实,结论可靠,文献引述全面,是一篇优秀的博士学位论文,表明了该同学在本专业方向具有坚实的理论基础和熟练的实验技能,具独立科研的能力。答辩过程表述清晰,回答问题正确。经答辩委员会无记名投票,一致同意通过姚睿同学的博士学位答辩,并建议授予理学博士学位。



\end{resolution}
