% !TeX root = ./PhDThesis.tex

\chapter{Conclusion and outlook}

To sum up, we have shown the main processes and techniques used to build a quantum memory based on trapped ions, including the experimental setup, the cryogenic experimental techniques, stable trapping and experimental realization of quantum memory.

The rate of background gas collisions with the ion chain is one of the scaling challenges in an ion-trapped system. Thus, we have constructed an extreme high vacuum environment to reduce pressure of the background gas in this vacuum system. For a newly built cryogenic trap system, the experimental implementation of stable trapping of multiple ions may take most of the project time before some quantum computing or quantum simulation experiments can be carried out. Therefore we spent a lot of time getting stable trapping of 200 ions. In terms of experimental techniques, we have achieved high fidelity initial state preparation and final state detection (the EMCCD-based spatially resolvable detection system allows for minimal crosstalk between different ions with a fidelity of 97\%), analyzed and suppressed the sources of noise from mechanical and electrical systems that induce decoherence. Finally, we report the stable trapping of above 200 ions in a zig-zag structure, and demonstrate the combination of the multi-qubit capacity and long storage lifetime by measuring the coherence time of several arbitrarily chosen ions to be on the order of hundreds of milliseconds. We compare the performance of the quantum memory with and without the sympathetic cooling laser, thus unambiguously show the necessity of sympathetic cooling for the long-time storage of multiple ionic qubits.

In the future, we can expect to upgrade various aspects of the cryostat design. To improve the scalability of the system, the implementation of a 2D ion crystal is a brilliant idea and one can try to combine monolithic trap and cryostat. Optimising the structure of the cryostat can further suppress vibration noise. Using laser ablation loading instead of oven loading can help to reduce the temperature and pressure of the background gas. The introduction of individual addressing into cryogenic trap systems can implement high fidelity quantum gate operations between arbitrary two qubits. Developments in cryogenic trap technology could be the future of trapped ion quantum computing.
