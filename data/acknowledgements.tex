% !TeX root = ./PhDThesis.tex

\begin{acknowledgements}
  始于初秋,终于盛夏. 每一段学习的旅程好像都是这样,在秋高气爽的九月入学,在烈日炎炎的六月毕业。清华大学给我提供了世界一流的学习科研平台,我有幸在这里度过了五年的难忘时光,期间我得到了老师、家人、同学、朋友的关怀和帮助。在此我向他们表示我最诚挚的谢意。国力日渐强盛是科技进步的保障,在此我向众志成城,共同战胜新冠疫情的中国人民致以崇高的敬意。

  衷心感谢导师段路明教授对我的精心指导,他的言传身教将使我终生受益。段路明教授是一位理论物理学大师:他完成了量子信息领域一些开创性的工作,提出实现长距离量子通信网络的量子中继方案,被国际同行誉为“DLCZ”方案,为该领域的奠基性方案;他提出通过量子网络互联进行规模化量子计算的方案,为近期离子量子计算的规模化发展奠定了理论基础。段路明教授还是一位实验物理学大师,目前超导和离子阱量子计算被广泛认为是最有可能实现通用量子计算的物理体系,此外在光量子、NV色心、冷原子等体系上的研究也取得了突破性进展,段路明教授对于所有这些领域都有深刻理解,包括理论上和实验上的深刻理解。确立正确的研究方向是科研工作的重点,得益于理论知识的融会贯通和实验知识的全面了解,段路明教授带领整个团队一直走在全球科研领域的前列,例如提出搭建低温离子阱系统、开发独立寻址的光学系统、使用同种离子作为大规模量子计算的物理体系等。强大的执行力是段路明教授给我的另一印象,物理学是一门实验科学,段路明教授总是能产生许多令人惊讶的好主意,然后组织起合适的团队进行仔细的求证,并及时地召集大家讨论、总结、提高。在和段路明教授的交流中,我能感受到他是一位谦虚、随和、令人尊敬的老师和朋友。在学习和科研中,段路明教授以身作则,给我树立了精益求精、一丝不苟的好榜样。做物理实验有时需要认真,有时需要灵活,还有时需要一些运气,但是这些都离不开方法。在我搭建这套低温离子阱系统期间,段路明教授指导我去学习做实验的基本原则,包括实验安全、实验规范等,养成一些做实验的好习惯,包括进行规范的实验记录、及时地完成实验总结、认识并掌握与他人进行学术交流的必要性和方法。在一段时间的基础学习后,我逐渐接手更复杂的实验课题,在这期间段路明教授给我最多的是鼓励,这些关怀对于进行需要不断试错的前沿物理实验的我来说是至关重要的。每周固定进行的组会上段路明教授和我会进行关于实验进展的讨论,这是推动一个课题前进的基石。每当我遇到复杂问题,找不到头绪的时候,段路明教授总能在简短的交流后抓住问题的核心。学会抓住问题的核心、了解并对比同行遇到的困难、主动和别人交流并毫不吝啬地提供建议,诸如此类的做人做事的方法不胜枚举。能有这样的学习的机会,我感到非常幸运。

  交叉信息研究院的何丽老师、周子超老师、吴宇恺老师给我的学习和生活提供了非常多的指导和帮助。何丽老师是一位认真负责、和蔼可亲、经验丰富的前辈,在我开始做实验的时期手把手教学,而我作为一个新人经常会出现失误,她会耐心指导、给予鼓励,对于新问题她总能引导我阐述自己的思考并与我交流她的独特见解,这些指导和帮助对于我的成长非常重要。周子超老师是一位认真负责、和蔼可亲、以身作则的前辈,他有很强的执行力,能够推动整个团队按部就班地展开工作,给我提供了良好的科研环境,同时他在工作上一丝不苟、以身作则,给我树立了榜样。吴宇恺老师是一位年轻有为、知识渊博的理论物理学家,同时他对于实验物理的了解也广泛且深刻,每当我有棘手的理论或实验问题,总是会在与他讨论的过程中受到启发,在科研论文写作中他给与我极大的支持,聪明且努力还要做到认真,这是我从他身上学到的。主管科研工作的常秀英老师、姚麟老师、祁宾祥老师、黄园园老师,他们在科研上给予了我很大的帮助。主管行政工作的尚妤婵老师、马佳老师、郭硕老师、孙帅老师,他们在学习生活上给予了我很大的帮助。

  我的父母是我坚强的后盾,在新冠疫情期间,他们给予了我无微不至的照顾,在我遇到挫折的时候,他们用陪伴给与了我最好的鼓励。

  在实验室和我一起工作学习的学长学姐学弟学妹们是我的好榜样和好朋友。在向连文倩学习做实验的过程中,我学到了如何快速推进实验和抓住实验重点。


\end{acknowledgements}
