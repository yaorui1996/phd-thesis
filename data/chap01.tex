% !TeX root = ./PhDThesis.tex

\chapter{Introduction}

研究生学位论文撰写,除表达形式上需要符合一定的格式要求外,内容方面上也要遵循一些共性原则。

通常研究生学位论文只能有一个主题(不能是几块工作拼凑在一起),该主题应针对某学科领域中的一个具体问题展开深入、系统的研究,并得出有价值的研究结论。
学位论文的研究主题切忌过大,例如,“中国国有企业改制问题研究”这样的研究主题过大,因为“国企改制”涉及的问题范围太广,很难在一本研究生学位论文中完全研究透彻。



\section{论文的语言及表述}

除国际研究生外,学位论文一律须用汉语书写。
学位论文应当用规范汉字进行撰写,除古汉语研究中涉及的古文字和参考文献中引用的外文文献之外,均采用简体汉字撰写。

国际研究生一般应以中文或英文书写学位论文,格式要求同上。
论文须用中文封面。

研究生学位论文是学术作品,因此其表述要严谨简明,重点突出,专业常识应简写或不写,做到立论正确、数据可靠、说明透彻、推理严谨、文字凝练、层次分明,避免使用文学性质的或带感情色彩的非学术性语言。

论文中如出现一个非通用性的新名词、新术语或新概念,需随即解释清楚。
