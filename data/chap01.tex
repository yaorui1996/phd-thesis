% !TeX root = ./PhDThesis.tex

\chapter{Introduction}

\section{Trapped-ion system}

During the 1950s, Wolfgang Paul and Hans Dehmelt have made significant contributions to the invention and development of the ion trap technology. The first experiment on ion and atom trapping was carried out by Paul's group in Bonn, Germany. They demonstrated the use of a six-pole magnetic field to confine atoms in a beam. Subsequently, they demonstrated that ions with various masses could be separated while passing through this device by utilizing four-pole electrodes with a radio-frequency field overlaid. Since then, the Paul trap \cite{RevModPhys.62.531} has evolved into a mass filter and is utilized extensively in a variety of fields, including mass spectroscopy and ultra-high vacuum pumps.

Trapped ions platform is a promising candidate for quantum computing and quantum simulations for its long coherence time, high fidelity qubit state detection and entanglement \cite{RN297,RN125}. Despite all of these achievements in this field, there is still much room for development. Scalability remains a difficulty. When the number of qubits rises, high-delity detection may be degraded by detection crosstalk, and entanglement techniques may become more complicated, particularly in quantum computing \cite{RN125}. The quantum information is encoded in a chain of qubits, each of which is a two-level system of the internal states of an ion that has been trapped. Primarily, there are two types of qubits. As a qubit, a meta-stable excited state and a ground state are selected for the optical qubit. And the other is the hyperfine qubit, which is comprised of two hyperfine levels in the ground-state manifold. The relaxation time $T_1$ of a hyperfine qubit is unlimited, and the anticipated coherence time $T_2$ might be greater than one hour utilizing the dynamic decoupling approach. In our lab, we concentrate on manipulations with ${ }^{171} \mathrm{Yb}^{+}$ ions, therefore the hyperfine qubit is often employed.



\section{Quantum computer}

The potential capabilities of quantum computers \cite{Benioff1980,Feynman1982} are now drawing a significant deal of research attention from the scientific community. As a result, the study of quantum computing has become a central topic of contemporary physics with a vast array of applications. To create universal quantum computing, numerous technologies have been suggested, including NMR, trapped ions, superconducting circuits, NV core, quantum dots, topological systems, etc \cite{Ladd2010}. Each has its own benefits and drawbacks. For instance, superconducting circuits benefit greatly from contemporary micro fabrication technologies, which makes superconducting qubits the most acknowledged by electrical engineers. The coherence duration for trapped ions may be as long as an hour, and the gate can be operated with the utmost precision, but at a significantly slower rate. Topological qubits are more difficult to fabricate, but theoretically, the requirements for quantum computing may be significantly reduced. Topological qubits also have inherent benefits for quantum error correction.

Trapped ions and superconducting circuits are now the most promising possibilities in development. There have been several demonstrations involving tens of qubits, and universal gates can be built with a high degree of accuracy over the threshold necessary for error correction. Especially, quantum superiority is proved by certain NP problems, such as the boson sampling, and numerous devices, including superconducting qubits and photonic qubits, are now capable of exhibiting the potential of quantum computing beyond conventional computers.


\section{Quantum memory}

The quantum memory is an important building block in quantum technology \cite{lvovsky2009optical}. For long-distance quantum communication and quantum cryptography, it lies at the core of the quantum repeater protocol which has an exponential improvement in the communication efficiency \cite{duan2001long, sangouard2011quantum}. For quantum computation, it synchronizes the qubits by appending identity gates between different quantum operations, and it allows the preparation of ancilla states in advance, which comprises the major cost of the fault-tolerant quantum computing \cite{gottesman1998theory,campbell2017roads}. Furthermore, matter qubits like trapped ions and superconducting circuits \cite{huang2020superconducting} by themselves can be regarded as quantum memories, whose performance fundamentally bounds those of all the quantum operations on these qubits.

Several figures of merit are used to characterize a quantum memory, such as its storage delity, lifetime and capacity \cite{lvovsky2009optical}. For applications in quantum networks, conversion delity and efficiency between matter qubits and photonic qubits are also concerned. For example, atomic ensembles have demonstrated single-excitation storage lifetime around 0.2 s \cite{yang2016efficient}, single-qubit storage delity of 99\% and efficiency of 85\% \cite{wang2019efficient}, and storage capacity of 105 qubits respectively in individual experiments; solid-state spins based on rare-earth-doped crystals have achieved storage lifetime of tens of milliseconds \cite{ortu2022storage} and the capacity of tens of temporal modes \cite{lago2021telecom}, and a spin ensemble coherence time above 6 hours \cite{zhong2015optically}; the NV center in a diamond has also realized 10-qubit storage for a lifetime above one minute \cite{bradley2019ten}. As one of the leading quantum information processing platforms, trapped ions keep the record for the longest single-qubit storage life-time above one hour \cite{wang2021single}. Entanglement between ionic and photonic qubits has also been demonstrated \cite{blinov2004observation,stute2012tunable,hucul2015modular,bock2018high,krutyanskiy2019light} as a plausible way to scale up the ion trap quantum computer \cite{duan2010colloquium,hucul2015modular}. As for the multi-qubit storage capacity, above 100 ions have been trapped in one-dimensional (1D) structure with spatial resolution \cite{pagano2018cryogenic}, and the global operation and the individual readout of above 60 ions have also been achieved \cite{li2023probing}. Shallow-depth quantum circuits composed of high-delity single-qubit and two-qubit gates have been realized for tens of ions \cite{egan2021fault,postler2022demonstration}, which provide an upper bound on the noise for a storage time below milliseconds.

Various laser cooling methods have been used to suppress the thermal motion of the ions such as the Doppler cooling \cite{leibfried2003quantum}, electromagnetic-induced-transparency (EIT) cooling \cite{qiao2021double, feng2020efficient}, polarization gradient cooling \cite{joshi2020polarization}, and sideband cooling \cite{leibfried2003quantum}.. However, all these cooling mechanisms rely on the transition between internal levels of the ions, thus will destroy the stored quantum information. Previously for the single-ion quantum memory, sympathetic cooling through a different ion species has been exploited \cite{wang2021single,chen2017sympathetic, blinov2002sympathetic, barrett2003sympathetic}. For longer ion chains, sympathetic cooling has also been achieved via focused laser beams on an optimized small fraction of ions \cite{mao2021experimental}. Here we take this approach to maintain the stability of 218 ions in a quasi-1D zigzag structure by sympathetically cooling the ions in the middle, and demonstrate that the arbitrarily chosen storage ions on the edges can achieve a typical storage lifetime above hundreds of milliseconds. Our work thus showcases the multi-qubit storage capacity and the long storage lifetime of the quasi-1D structure of the trapped ions, which can find applications in deep quantum computing circuits and quantum networks.
