% !TeX root = ./PhDThesis.tex

\chapter{Experimental Procedure}

\section{Start CryoServer}

\section{Cooling-down and warm-up}

The cryogenic trapped-ion system is a relatively complex experimental system, and we need the system to be stable over a long period of time so that the reproducibility of the measurement results is high. Although the cryostat's core component, the cold head, can run continuously for more than XX hours, the maximum time this cryostat can run continuously is limited by the stability of the power supply, the stability of the laboratory temperature and humidity, and whether the exchange gas chamber is leaking. It took us about three years to get the system into a stable long-term state, after which we conducted a series of physical experiments on the experimental platform. However, during the three-year commissioning process, we inevitably need to conduct the cycle of cooling-down, malfunction, warm-up, and upgrade, during which the standardized operation helps to make the physical parameters of the system more repeatable, so we have developed a standardized operation procedure for this system.

\subsection{Maintenance of the exchange gas chamber}

If the cold head does not need to be removed for servicing, the exchange gas chamber does not require frequent maintenance and is always in an independent and stable state, whether it is being cooled down or warmed up.

The exchange gas chamber uses helium gas with a purity of XX. When we expose the exchange gas chamber to atmosphere or when it is first used, the internal gas needs to be purified. According to the cryostat manufacturer's recommendations, a purification is also required after several months of continuous running, but this is not normally done when the system is stable for a long period of time. How often the exchange gas chamber needs to be purified depends on the rate of impurity gases (nitrogen, oxygen, water vapour etc.) leaking in from the atmosphere.

When we need to purify the helium gas in the exchange gas chamber, the exchange gas chamber is first evacuated continuously for 0.5 hours with a dry scroll pump (Agilent IDP-7), then the valve connected to the dry scroll pump is closed and the valve connected to the helium gas is opened. The auto gas charging system will then raise the pressure to 1.03 bar and finally we close the valve to the helium gas. In general, the above operation is repeated three times to purify the helium gas in the exchange gas chamber.

When we need to cool down or warm up the system, and also when the system is running at low temperatures for a long time, we simply open the valve to the helium gas and keep the auto gas charging system running steadily.

\subsection{Cooling-down}

In the Cooling-down procedure, the physical parameters of the vacuum chamber are mainly adjusted and observed. The vacuum chamber is first connected to a turbo-molecular pump (TPS-compact Turbo Pumping System) via the angle valve and after approximately 48 hours of continuous operation the vacuum chamber reaches a vacuum level close to UHV. The ion gauge is switched on and reaches an indication of 5E-8 mBar, at which point we do not need to degas the ion gauge as the room temperature zone of the vacuum chamber does not eventually fall below 1E-10 mbar. Now we need to perform a time limited activation of the NEG Pump for 1 hour, then we perform several degas of the Ion Pump and keep the Ion Pump on. Now that the activation of the NEG-Ion Pump is complete, we close the angle valve and wait about 1 hour for the ion gauge to gradually decrease to 3E-9 mbar, when the vacuum chamber reaches the UHV vacuum level. We turn on the cold head and the temperature stabilize system, which will finish cooling down within 5 hours, but the system will not reach final stabilization for more than 24 hours. The temperature of the 4K stage is finally stabilised at 6K and the ion gauge is stabilised at 3E-10mbar.

\subsection{Warm-up}

The warm-up procedure is much easier than the cooling-down procedure because we do not need to obtain UHV during this process. we turn off the cryogenic and vacuum related instruments: the NEG-Ion pump, the ion gauge, the cold head. We can use the heater in the temperature stabilize system to heat the cryostat to speed up the warming process to room temperature, which takes about 24 hours or more. The system can also be allowed to warm up naturally to room temperature, which takes about 48 hours or more. Next, if necessary, we can move the cryostat into the service area for servicing. Before moving it out, we need to record the readings of all optical and electrical instruments. As the imaging system is embedded in the re-entrant window, we usually need to remove the objective lens.
