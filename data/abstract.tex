% !TeX root = ./PhDThesis.tex

% 中英文摘要和关键字

\begin{abstract}

  量子存储器是量子技术的一个重要组成部分,它在量子计算机和量子网络中得到了广泛的应用,而存储寿命和容量是表征其性能的两个重要因素,这项工作展示了如何提升这些性能。得益于控制性与扩展性,离子阱系统展现出了其利于量子计算和量子存储的能力,并且被认为是最先进的平台之一。

  我们设计、搭建并检验了这套离子阱系统。这个物理装置的性能达到了世界一流水平,表现在:其常温区真空度达到了$3 \times {10}^{-10}$ mbar,能够实现多离子稳定囚禁,实验数据证明对于离子数超过200的一维离子链能够囚禁24小时以上,通过测量多离子zig-zag构型的寿命等方法可以估算出低温区域的真空度低于$1 \times {10}^{-12}$ mbar;精巧的机械和光学设计使其具有进行量子计算的潜力,振动和电学噪声得到很大的抑制;使用协同冷却技术极大地提升了其作为量子存储器的存储寿命和容量。在这套系统中,我们探索并实现了一种进行多离子稳定囚禁的方法,并将整个过程标准化使其成功经验能够被复制;在这套标准化的流程中,我们将重要环节精细化,包括真空腔体的处理、获取超高真空的流程、光学模块的搭建、电子控制模块中硬件系统和软件系统的优化;同时我们也实现了离子囚禁和量子操控的自动化,得益于此我们可以进一步探索多离子稳定囚禁的上限。在实验技术上,我们实现了高保真度的初态制备和末态探测(基于EMCCD的空间可分辨的探测系统使得不同离子之间的串扰很小,保真度达到97\%),分析并抑制了来自于机械和电学系统的造成退相干的噪声来源。

  利用这个离子阱系统,我们实现了在zig-zag构型中稳定地捕获200个以上的离子,并通过测量几个任意选择的离子的相干时间,证明了多量子比特容量和长存储寿命的结合,其相干时间为数百毫秒。我们比较了有无协同冷却激光下的量子存储器的性能,从而毫不含糊地表明协同冷却对于长时间存储多个离子量子比特的必要性。

  通过这项工作,我们展现了离子阱系统可以作为一种强大的量子存储器,其长时间连贯存储大量量子比特的能力对于未来具有深度电路的量子计算任务来说是至关重要的。

  % 关键词用“英文逗号”分隔,输出时会自动处理为正确的分隔符
  \thusetup{
    keywords = {离子阱, 冷阱, 量子计算, 量子存储器},
  }
\end{abstract}

\begin{abstract*}

  The quantum memory is an important building block in quantum technology, which is widely used in quantum computing and quantum networks, and storage lifetime and capacity are two important factors to characterize the performance of a quantum memory, this work shows how these performances can be improved. Owing to its controllability and scalability, the trapped ion system has demonstrated its ability to facilitate quantum computing and quantum memory, and is considered to be one of the most advanced platforms.

  We have designed, built and tested this trapped ion system. The performance of this physical device is state-of-the-art, as evidenced by: a pressure of $3 \times {10}^{-10}$ mbar in the room temperature region, the ability to achieve multi-ion stable trapping, experimental data demonstrating that one-dimensional ion chains with ion numbers over 200 can be trapped for more than 24 hours, the pressure in the cryogenic region can be estimated to be below $1 \times {10}^{-12}$ mbar, for example by measuring the lifetime of the multi-ion zig-zag configuration; the delicate mechanical and optical design provides the capability for quantum computing, with significant suppression of vibration and electrical noise; the use of sympathetic cooling technique greatly enhances its storage lifetime and capacity as a quantum memory. In this system, we have explored and implemented a method for multi-ion stable trapping and standardized the process so that this device is abled to be replicated; in this standardized process, we have refined important aspects, including the handling of the vacuum chamber, the process of obtaining ultra-high vacuum, the construction of the optical module, and the optimization of the hardware and software systems in the electronic control module; we have also automated the process of ion trapping and quantum manipulation, thanks to which we can further explore the upper limits of stable trapping of multiple ions.

  In terms of experimental techniques, we have achieved high fidelity initial state preparation and final state detection (the EMCCD-based spatially resolvable detection system allows for minimal crosstalk between different ions with a fidelity of 97\%), analyzed and suppressed the sources of noise from mechanical and electrical systems that induce decoherence.

  Here we report the stable trapping of above 200 ions in a zig-zag structure, and demonstrate the combination of the multi-qubit capacity and long storage lifetime by measuring the coherence time of several arbitrarily chosen ions to be on the order of hundreds of milliseconds. We compare the performance of the quantum memory with and without the sympathetic cooling laser, thus unambiguously show the necessity of sympathetic cooling for the long-time storage of multiple ionic qubits.

  With this work, we show that trapped ion systems can act as a powerful quantum memory, and that their capability to coherently store a large number of qubits for long time is crucial for the quantum computing tasks in the future with deep circuit depth.

  % Use comma as separator when inputting
  \thusetup{
    keywords* = {Trapped Ions, Cryogenic Trap, Quantum Computing, Quantum Memory},
  }
\end{abstract*}
